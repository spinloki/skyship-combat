\newcommand{\OutflankedSidebar}{
\begin{tcolorbox}[
    enhanced,
    sidebyside,
	float,
    sidebyside align=top seam,
    width=\textwidth,
    colback=gray!10,
    colframe=black,
    fonttitle=\bfseries,
    title=Outflanked,
    lefthand width=0.58\textwidth,
    righthand width=0.38\textwidth,
    sharp corners=southwest,
    breakable
]
If a ship is \outflanked, the \defender{} can no longer bid \dicetype{\maneuver} against the \attacker{} for the rest of the round.
\hfill \break \break
Additionally, in the ensuing Shooting Action, the \attacker{} may choose \textbf{two} of the following:
\begin{itemize}[noitemsep]
    \item The \attacker{} may choose the \defender{}'s facing.
	\item The \defender{} cannot gain any \dicetype{\attack}.
    \item The \defender{} cannot gain any \dicetype{\evasion}.
\end{itemize}
\end{tcolorbox}
}

\newcommand{\IgnoreDiceSidebar}{
\begin{tcolorbox}[
    enhanced,
    sidebyside,
	float,
    sidebyside align=top seam,
    width=\textwidth,
    colback=gray!10,
    colframe=black,
    fonttitle=\bfseries,
    title=Ignoring Area Denial and Escort Dice,
    lefthand width=0.58\textwidth,
    righthand width=0.38\textwidth,
    sharp corners=southwest,
    breakable
]
During the Maneuver Phase, \defender{}s may benefit from:

\begin{itemize}[noitemsep]
    \item \dicetype{\areadenial} — rolled from weapons
    \item \dicetype{\escort} — provided by allied ships
\end{itemize}

The \attacker{} may choose to \textbf{ignore} these dice after they are rolled:

\begin{itemize}[noitemsep]
    \item If an \dietype{\areadenial} is ignored, the \defender{} loses its benefit, but the \attacker{} \textbf{takes a hit} from that weapon.
    \item If an \dietype{\escort} is ignored, the \defender{} gains no bonus -- but the \attacker{} is \outflanked{} by the escort once the \attacker{}'s Shooting Action is done.
\end{itemize}
\end{tcolorbox}
}

\newcommand{\FleeSidebar}{
\begin{tcolorbox}[
    enhanced,
    sidebyside,
	float,
    sidebyside align=top seam,
    width=\textwidth,
    colback=gray!10,
    colframe=black,
    fonttitle=\bfseries,
    title=Fleeing Combat,
    lefthand width=0.58\textwidth,
    righthand width=0.38\textwidth,
    sharp corners=southwest,
    breakable
]
Instead of bidding two dice to maneuver against another ship, a ship may bid \textbf{all} of its remaining maneuver dice to attempt to flee the combat.
\hfill \newline \newline
The fleeing ship can gain \dicetype{\areadenial{} or \escort}, the same as if it were the Defender in a Maneuver Action.
\hfill \newline \newline
All opposing ships can pool \dicetype{\maneuver} together to match the fleeing ship's bid. They can use \dicetype{\areadenial{} and \environmental} as normal.
\hfill \newline \newline
The fleeing ship escapes if the opposing side cannot match its bid. Otherwise, all bid dice are lost.
\end{tcolorbox}
}