\documentclass[11pt]{article}
\usepackage[margin=1in]{geometry}
\usepackage{titlesec}
\usepackage{enumitem}
\usepackage{booktabs}
\usepackage{fancyhdr}
\usepackage{lmodern}
\usepackage{sectsty}
\usepackage{multicol}
\usepackage{wrapfig}
\usepackage[most]{tcolorbox}
\usepackage{float}
\tcbuselibrary{listings, skins, breakable, theorems}
\usepackage{hyperref}
\hypersetup{
    colorlinks,
    citecolor=black,
    filecolor=black,
    linkcolor=black,
    urlcolor=black
}

\usepackage{xcolor}
\newcommand{\Power}{\textsc{Power}}
\newcommand{\Advantage}{\textbf{Advantage}}
\newcommand{\Outflanked}{\textit{Outflanked}}

\newcommand{\rangeband}[2]{\textbf{#1:} #2}
\newcommand{\ShipName}{\textsc{Vulture-class Interceptor}}

\newcommand{\Fire}{\textbf{\textcolor{red}{Fire!}}}

\newcommand{\maneuver}{\textbf{Maneuver}}
\newcommand{\evasion}{\textbf{Evasion}}
\newcommand{\speed}{\textbf{Speed}}
\newcommand{\turn}{\textbf{Turn}}
\newcommand{\attack}{\textbf{Attack}}
\newcommand{\escort}{\textbf{Escort}}
\newcommand{\areadenial}{\textbf{Area Denial}}
\newcommand{\environmental}{\textbf{Environmental}}

\newcommand{\dietype}[1]{\textbf{#1 Die}}
\newcommand{\dicetype}[1]{\textbf{#1 Dice}}

\newcommand{\outflanked}{\textbf{Outflanked}}
\newcommand{\advantage}{\textbf{Advantage}}

\newcommand{\attacker}{Attacker}
\newcommand{\defender}{Defender}

\newcommand{\attackingside}{Attacking Side}
\newcommand{\defendingside}{Defending Side}

\newcommand{\listentry}[2]{\textbf{#1:} #2}
\newcommand{\modulelistentry}[4]{\textbf{#1:} #2 \hfill{} \newline \textbf{Cost: #3 Wealth, #4 Power}}

\title{Skyship Combat System}
\author{Iravol System Reference}
\date{}

\pagestyle{fancy}
\fancyhf{}
\rhead{Iravol}
\lhead{Skyship Combat Rules}
\rfoot{\thepage}

\begin{document}

\maketitle
\tableofcontents
\newpage

\section{Overview}

Skyship combat in Iravol is divided into two distinct phases: the \textbf{Maneuver Phase} and the \textbf{Shooting Phase}. Combat emphasizes initiative, positioning, environmental exploitation, and resource management through dice pools and power usage. \ShipName

\section{Ship Construction Rules}

\subsection{Chassis Overview}

Each ship begins with a \textbf{Chassis}, which provides its fundamental structure and limitations. A Chassis defines the following core statistics:

\begin{itemize}
    \item \textbf{Core Modification Slots:} Number of integral modifications the ship can support (usually 1).
    \item \textbf{Hull Integrity:} Number of hits the ship can take before catastrophic failure.
    \item \textbf{Armor (per Facing):} Baseline armor value for each of the four ship facings.
    \item \textbf{Weapon Mounts (per Facing):} Mount points for Small, Medium, or Large weapons.
	\item \textbf{Speed and Turn rating:} Two components of the ship's base maneuver pool.
    \item \textbf{Power Rating:} The total amount of Power the ship generates and can spend per combat round.
\end{itemize}

\subsection{Ship Construction Process}

\begin{enumerate}
    \item \textbf{Choose a Chassis:} Select a base hull that fits your intended role (interceptor, cruiser, etc.).

    \item \textbf{Install Core Modifications:} Choose up to the allowed number of \textbf{Core Mods}, which alter ship behavior (e.g., Armored Core, Overdrive Matrix).
    
    \item \textbf{Mount Weapons:}
    \begin{itemize}
        \item Weapons are mounted by \textbf{Facing} (Front, Rear, Port, Starboard).
        \item Each mount is designated as \textbf{Small}, \textbf{Medium}, or \textbf{Large}.
        \item Weapons have a size requirement and can only be mounted in appropriately sized slots.
        \item Each weapon may also have \textbf{one Weapon Modification} (e.g., Unstable Ammunition, Stabilized Targeting).
    \end{itemize}
    
    \item \textbf{Install Components:} Choose support systems (e.g., Engine Echo Projector, Flow-Tuned Rigging), limited only by the ship’s available \textbf{Power Rating}.
    
    \item \textbf{Validate Power Usage:} Ensure that the sum of all component and weapon \textbf{Power Costs} does not exceed the ship’s \textbf{Power Rating}.
\end{enumerate}

\subsection{Weapons and Range}

Each weapon defines its own effectiveness at different ranges, expressed as a dice pool per range band:

\begin{center}
\begin{tabular}{lccccc}
\toprule
\textbf{Weapon} & \textbf{PB} & \textbf{Short} & \textbf{Medium} & \textbf{Long} & \textbf{Extreme} \\
\midrule
Light Cannon & 3d6 & 3d6 & 2d6 & 1d6 & — \\
Sniper Lance & — & 1d6 & 2d8 & 3d8 & 4d6 \\
Spray Gun & 4d4 & 3d4 & 2d4 & — & — \\
\bottomrule
\end{tabular}
\end{center}

Range bands are determined during the Shooting Phase. Some weapons may include \textbf{range-shifting abilities} (e.g., Stabilized Targeting) or traits like \textbf{Spray}, \textbf{Area Denial}, or \textbf{Echo} that interact with this table.

\subsection{Power System Summary}

Each component and weapon contributes a fixed Power Cost to the ship. A ship’s \textbf{Power Rating} is the total amount of power it can spend per round. Power is used for:

\begin{itemize}
    \item Flaring Drives for bonus maneuver dice.
    \item Flaring Defense systems for evasion.
    \item Activating components (e.g., Stormcaller, Signal Scrambler).
    \item Triggering certain weapon mods or rerolls.
\end{itemize}

If a ship ends a round with \textbf{0 Power}, it suffers the \textbf{Drive Failure} critical effect.

\subsection{Weapon Mount Sizes}

\begin{itemize}
    \item \textbf{Small Mount:} Can fit light weapons, often used for defense or area denial. Usually 1–2 dice per range.
    \item \textbf{Medium Mount:} Standard offensive weapons. Balanced across range bands.
    \item \textbf{Large Mount:} Heavy weapons, often have limited arcs but high damage. May require more Power.
\end{itemize}

\subsection{Wealth and Cost}

The cost of weapons, components, and chassis upgrades is governed by a \textbf{Wealth System}, which is not covered in this document. Players and GMs should agree on available budget for outfitting based on campaign context.



\section{The Combat Round}

\subsection{Maneuver Actions}

\begin{enumerate}
    \item \textbf{Roll Maneuver Dice:} Each ship rolls its maneuver pool (speed + maneuverability + pilot skill).
    
    \item \textbf{Determine Initiative:} 
    \begin{itemize}
        \item Attacking side goes first.
        \item If both sides are attacking, the side with the highest single dice roll is treated as the Attacker.
    \end{itemize}

    \item \textbf{Claim Environmental Dice Pools:}
    \begin{itemize}
        \item Attacking side chooses one environmental pool.
        \item Defending side chooses from remaining options.
    \end{itemize}

    \item \textbf{Maneuver Actions:}
    \begin{enumerate}
        \item Attacker chooses a ship to act.
        \item Attacking ship may roll 1 die from an Area Denial weapon.
        \item Attacking ship bids two dice to maneuver against a target.
        \item Defender may:
        \begin{itemize}
            \item Roll Area Denial weapon (1 die).
            \item Receive escort (1 die per 2 sacrificed from an ally).
            \item Match attacker’s bid using any number of maneuver dice.
        \end{itemize}
        \item If the defender matches the bid using fewer dice, it gains \textbf{Advantage}.
        \item If the defender cannot match, it is \textbf{Outflanked}.
        \item Any unused Escort or Area Denial dice expire.
    \end{enumerate}

    \item Repeat this process for the Defender.
	\begin{enumerate}
		\item Either side can decline to act on their turn
		\item If both sides decline (or are unable) to act, all maneuver dice expire, and a new round begins.
	\end{enumerate}
\end{enumerate}

\subsection{Shooting Actions}

\begin{enumerate}
    \item \textbf{Determine Facing and Range:}
    \begin{itemize}
        \item The ship with Advantage chooses its own facing.
        \item The opposing ship chooses its own facing.
		\item The ship with Advantage chooses the engagement range.
    \end{itemize}

    \item \textbf{Firing and Defending:}
    \begin{itemize}
        \item Each ship may fire once from each weapon on the chosen facing.
        \item Shooter declares which weapons it will fire (forming the attack pool).
        \item Defender decides whether to Evade and/or fire Defensive weapons to gain Evasion Dice.
        \item Both sides roll their respective dice pools.
    \end{itemize}

    \item \textbf{Attack Resolution (Bid-Match):}
    \begin{enumerate}
        \item Shooter bids one weapon die.
        \item Defender must match or exceed using evasion/defense dice, or take the hit.
        \item Repeat until shooter is out of attack dice.
    \end{enumerate}

    \item \textbf{Resolve Hit Effects:}
    \begin{itemize}
        \item Subtract armor from each successful hit.
        \item Apply leftover damage using the Damage Consequences Table.
    \end{itemize}
\end{enumerate}

\begin{tcolorbox}[
    enhanced,
    sidebyside,
	float,
    sidebyside align=top seam,
    width=\textwidth,
    colback=gray!10,
    colframe=black,
    fonttitle=\bfseries,
    title=Outflanked,
    lefthand width=0.58\textwidth,
    righthand width=0.38\textwidth,
    sharp corners=southwest,
    breakable
]
If a ship is Outflanked, the Defending ship can no longer bid Maneuver Dice against the Attacking ship for the rest of the round.
Additionally, in the ensuing Shooting Action, the Attacker may choose \textbf{two} of the following:
\begin{itemize}[noitemsep]
    \item The Attacker may choose the Defender's facing.
	\item The Defender cannot gain any Attack dice.
    \item The Defender cannot gain any Evasion dice.
\end{itemize}
\end{tcolorbox}

\begin{tcolorbox}[
    enhanced,
    sidebyside,
	float,
    sidebyside align=top seam,
    width=\textwidth,
    colback=gray!10,
    colframe=black,
    fonttitle=\bfseries,
    title=Ignoring Area Denial and Escort Dice,
    lefthand width=0.58\textwidth,
    righthand width=0.38\textwidth,
    sharp corners=southwest,
    breakable
]
During the Maneuver Phase, defenders may benefit from:

\begin{itemize}[noitemsep]
    \item \textbf{Area Denial Dice} — rolled from weapons
    \item \textbf{Escort Dice} — provided by allied ships
\end{itemize}

The attacker may choose to \textbf{ignore} these dice after they are rolled:

\begin{itemize}[noitemsep]
    \item If an \textbf{Area Denial Die} is ignored, the defender loses its benefit, but the attacker \textbf{takes a hit} from that weapon.
    \item If an \textbf{Escort Die} is ignored, the defender gains no bonus -- but the escort \textbf{Outflanks the attacker} once the Attacker's Shooting Action is done.
\end{itemize}
\end{tcolorbox}

\newpage
\section{Damage Consequences Table}

\begin{tabular}{@{}ll@{}}
\toprule
\textbf{Post-Armor Damage} & \textbf{Effect} \\ \midrule
1--2 & Concussed: Reduce one maneuver die by damage dealt. \\
3--4 & Armor Breach: -1 armor to that facing (can go below 0). \\
5    & Fire! Facing ignites (see Fire rules). \\
6    & Component Damage: Attacker disables one known system. \\
7--8 & Drive Damaged: Target loses two maneuver dice permanently. \\
9--10 & Motive Failure: Target loses all base maneuver dice, and cannot regain them for the rest of the combat. \\
11+  & Ship Crippled: As above, and target can no longer attack. \\
\bottomrule
\end{tabular}

\section{Power System Overview}

Each ship has a limited Power Reserve, used to:
\begin{itemize}
    \item Flare Drives (gain bonus maneuver dice).
    \item Flare Defense (gain bonus evasion dice).
    \item Activate special components (e.g., Stormcaller).
    \item Use Overdrive or other modifications.
\end{itemize}
If a ship ends a round with 0 Power, it suffers \textbf{Motive Failure}.

\section{Weapon Qualities}

\begin{itemize}
    \item \textbf{Area Denial:} Weapon may roll one die during a maneuver to disrupt the enemy.
    \item \textbf{Spray:} At short ranges, attacker may combine multiple small dice for greater armor penetration.
    \item \textbf{Echo:} When a hit lands, resolve another hit at 2 less damage, repeat until zero.
    \item \textit{(More to be added...)}
\end{itemize}

\section{Support Components} % Placeholder
\begin{itemize}
    \item \textbf{Engine Echo Projector:} Placeholder
    \item \textbf{Flow-Tuned Rigging:} Placeholder
    \item \textit{(More entries forthcoming)}
\end{itemize}

\section{Hull Types} % Placeholder
\begin{itemize}
    \item \textbf{Vulture-class Interceptor:} Placeholder
    \item \textbf{Stormcaller Skiff:} Placeholder
    \item \item \textbf{Battleship Frame III:} Placeholder
\end{itemize}

\section{Weapons} % Placeholder
\begin{itemize}
    \item \textbf{Aether Cannon:} Placeholder
    \item \textbf{Mine Layer:} Placeholder
    \item \textbf{Spray Cannon:} Placeholder
\end{itemize}

\end{document}
