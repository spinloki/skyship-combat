\section{The Combat Round}

\subsection{Maneuver Actions}

\begin{enumerate}
    \item \textbf{Roll \dicetype{\maneuver}:} Each ship rolls its base \dicetype{\maneuver} pool (speed + maneuverability + pilot skill).
    
    \item \textbf{Determine Initiative:} 
    \begin{itemize}
        \item The \attackingside{} goes first.
        \item If both sides are attacking, the side with the ship that has the highest sum of \dicetype{\maneuver} is treated as the \attackingside.
    \end{itemize}

    \item \textbf{Claim \dicetype{\environmental} Pools:}
    \begin{enumerate}
        \item \attackingside{} chooses one \dicetype{\environmental} pool.
        \item \defendingside{} chooses one from remaining options.
    \end{enumerate}

    \item \textbf{Maneuver Actions:}
    \begin{enumerate}
        \item \attackingside{} chooses a ship to act. This is the \textbf{\attacker}.
        \item \attacker{} may roll 1 \dietype{\areadenial}, adding it to its \dicetype{\maneuver} pool.
        \item \attacker{} bids two \dicetype{\maneuver} against a target. This is the \textbf{\defender}.
        \item The \defender{} may:
        \begin{itemize}
            \item Roll 1 \dietype{\areadenial}, adding it to its \dicetype{\maneuver} pool.
            \item Receive \dicetype{\escort}. Allied ships may sacrifice two of their own \dicetype{\maneuver} to grant one of them to the \defender{}.
            \item Match \attacker{}’s bid using any number of \dicetype{\maneuver}.
        \end{itemize}
        \item Normally, the \attacker{} has \advantage. If the \defender{} matches with only one die, it can either take \advantage{} or designate an allied ship as the new \defender{}.
        \item If the \defender{} cannot match the \attacker{}'s bid, it is \outflanked.
        \item Any unused \dicetype{\escort{} or \areadenial} expire.
		\item Resolve the Shooting Action between the \attacker{} and \defender{}.
    \end{enumerate}

    \item Repeat this process for the \defendingside{}.
	\begin{itemize}
		\item Either side can decline to act on their turn.
		\item If neither side initiates a maneuver, all maneuver dice expire, and a new round begins.
		\item If the \attackingside{} initiates no maneuvers in a round, the \defendingside{} may choose to become the \attackingside{} in the next round.
		\item If neither side initiates maneuvers over the course of a round, either side may choose to end the combat.
	\end{itemize}
\end{enumerate}

\FleeSidebar

\subsection{Shooting Actions}

\begin{enumerate}
    \item \textbf{Determine Facing and Range:}
    \begin{enumerate}
        \item The ship with \advantage{} chooses its own facing.
        \item The opposing ship chooses its own facing.
		\item The ship with \advantage{} chooses the engagement range.
    \end{enumerate}

    \item \textbf{Firing and Defending:}
    \begin{enumerate}
        \item Each ship may fire once from each weapon on the chosen facing.
        \item Shooter declares which weapons it will fire (forming the \dicetype{\attack} pool).
        \item \defender{} decides whether to spend power to Evade, choosing to either gain \dicetype{\evasion} equal to its base \dicetype{\maneuver} pool or adjust the engagement range by one band.
        \item \defender{} decides whether to fire Defensive weapons to gain \dicetype{\evasion}.
        \item Both sides roll their respective dice pools.
    \end{enumerate}

    \item \textbf{Attack Resolution (Bid-Match):}
    \begin{enumerate}
        \item Shooter bids one weapon die.
        \item \defender{} must match or exceed using \dicetype{\evasion}, or take the hit.
        \item Repeat until shooter is out of \dicetype{\attack}.
        \item Repeat Attack Resolution, swapping \attacker{} and \defender.
    \end{enumerate}

    \item \textbf{Resolve Hit Effects:}
    \begin{enumerate}
        \item Subtract armor from each successful hit.
        \item Remaining damage up to armor is Hull Integrity damage. Anything beyond that is a Critical Hit.
        \item Apply the Critical Hit using the Critical Table. \attacker can choose a lower result, using the same damage amount.
		\item Suffer Hull Breached if Hull Integrity is reduced to 0.
    \end{enumerate}
\end{enumerate}

\OutflankedSidebar
\IgnoreDiceSidebar

\newpage

\subsection{Critical Table}

\begin{tabular}{@{}ll@{}}
\toprule
\textbf{Result} & \textbf{Effect} \\ \midrule
0 & Hull Ruptured: Target loses (Damage) additional Hull Integrity. \\
1 & Concussed: Reduce one maneuver die by (Damage). \\
2 & Armor Breach: Attack strips (Damage) armor from that facing. \\
3 & Component Damage: \attacker{} disables one known component. \\
4 & Motive Disruption: Target loses two base \dietype{\maneuver} and cannot regain them. \\
5 & Hull Breached: Armor on all facings set to 0. All Attack Dice can be combined against it. \\
6 & Motive Failure: Target loses all base \dicetype{\maneuver} and cannot regain them. \\
7+ & Ship Crippled: As above, and target can no longer gain \dicetype{\attack{} or \maneuver}. \\
\bottomrule
\end{tabular}


\subsection{Ship Power}

Each ship has a limited Power Reserve, depleted in the following ways:
\begin{itemize}
    \item Passive combat stress. Usually at a rate of 1 Power every 2 rounds at Halflight altitude.
    \item Flaring Drives to Evade during a Shooting Action, or to reroll Maneuver Dice at the start of a Combat Round.
    \item Activating special components.
\end{itemize}
If a ship ends a round with 0 Power, it suffers \textbf{Motive Failure}.
\hfill \newline
At \textbf{any time}, a ship can choose to disable one of its own components to restore 1 Power.